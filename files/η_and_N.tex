\documentclass{article}
\usepackage{geometry}
\geometry{a4paper, left=2.5cm, right=2.5cm, top=2.5cm, bottom=2.5cm}
\linespread{1.15} % 1.15倍行距,阅读更舒适
\usepackage{graphicx} % Required for inserting images
\usepackage{ctex}

% 1. 颜色包必须最先加载,并开启 dvipsnames 选项
\usepackage[dvipsnames]{xcolor}

% 2. 基础数学包
\usepackage{amsthm}
\usepackage{amsmath}
\usepackage{amssymb}

% 3. 定理工具包 + 边框支持包 (这里必须加载 mdframed)
\usepackage{thmtools}
\usepackage[framemethod=TikZ]{mdframed}

% 4. 定义样式
% 蓝色系样式(用于定义、假设)
\declaretheoremstyle[
    headfont=\bfseries\sffamily\color{NavyBlue}, % 标题字体颜色
    bodyfont=\normalfont,
    mdframed={
        linewidth=2pt,
        rightline=false, topline=false, bottomline=false, % 只保留左边线
        linecolor=NavyBlue,      % 左边线颜色
        backgroundcolor=NavyBlue!5, % 背景色 (5%透明度)
        innerleftmargin=10pt,    % 左内边距
        innerrightmargin=10pt,   % 右内边距
        innertopmargin=10pt,     % 上内边距
        innerbottommargin=10pt,  % 下内边距
        skipabove=10pt,          % 与上方文本间距
        skipbelow=10pt,          % 与下方文本间距
    }
]{thmblue}

% 红色系样式(用于定理、引理)
\declaretheoremstyle[
    headfont=\bfseries\sffamily\color{RawSienna},
    bodyfont=\normalfont,
    mdframed={
        linewidth=2pt,
        rightline=false, topline=false, bottomline=false,
        linecolor=RawSienna,
        backgroundcolor=RawSienna!5,
        innerleftmargin=10pt,
        innerrightmargin=10pt,
        innertopmargin=10pt,
        innerbottommargin=10pt,
        skipabove=10pt,
        skipbelow=10pt,
    }
]{thmred}
% remark
\declaretheoremstyle[
    headfont=\bfseries\sffamily\color{ForestGreen}, % 标题绿色
    bodyfont=\normalfont,
    mdframed={
        linewidth=2pt,
        rightline=false, topline=false, bottomline=false,
        linecolor=ForestGreen,       % 边框绿色
        backgroundcolor=ForestGreen!5, % 背景极淡的绿色
        innerleftmargin=10pt,
        innerrightmargin=10pt,
        innertopmargin=10pt,
        innerbottommargin=10pt,
        skipabove=10pt,
        skipbelow=10pt,
    }
]{thmgreen}

% 1. 加载 hyperref
\usepackage[
    colorlinks=true,
    linkcolor=NavyBlue,
    citecolor=ForestGreen,
    urlcolor=NavyBlue,
    pdfborder={0 0 0}
]{hyperref}

% 2. 加载 cleveref (必须在 hyperref 之后)
\usepackage[nameinlink]{cleveref}
\crefname{definition}{定义}{定义}
\crefname{assumption}{假设}{假设}
\crefname{theorem}{定理}{定理}
\crefname{lemma}{引理}{引理}
\crefname{proposition}{命题}{命题}
\crefname{corollary}{推论}{推论}
\crefname{remark}{Remark}{Remarks}

% 3. 最后执行 \declaretheorem (此时它能检测到 hyperref 并正确设置锚点)
\declaretheorem[style=thmblue, name=定义, numberwithin=section]{definition}
\declaretheorem[style=thmblue, name=假设, numberwithin=section]{assumption}

\declaretheorem[style=thmred, name=定理, numberwithin=section]{theorem}
\declaretheorem[style=thmred, name=引理, numberwithin=section]{lemma}
\declaretheorem[style=thmred, name=推论, numberwithin=section]{corollary}
\declaretheorem[style=thmred, name=命题, numberwithin=section]{proposition}

\declaretheorem[style=thmgreen, name=Remark, numberwithin=section]{remark}

\title{基于 Modes--Zipf--隐式偏置框架的\\学习率随模型规模缩放规律推导}
\author{Jiaxuan Zou}
\date{\today}

\begin{document}
\maketitle

\begin{abstract}
本文在一种可解析的 ''Modes--Zipf--隐式偏置'' 抽象框架下,给出学习率 $\eta$ 随模型规模 $N$ 的缩放规律推导。框架的核心是:数据由一族可数的 ''模式''(modes)组成,出现频率满足 Zipf 幂律;训练过程在每个模式上的剩余误差 $q_k$ 呈乘法收缩;并允许 ''频率调制'' 的有效学习强度 $\lambda_k$。在该框架中,训练动力学只通过内禀时间尺度 $s=c\tau$ 进入,其中 $c=\eta\lambda_0$、$\tau$ 为 token-step。要把学习率与模型规模关联起来,必须补充一条把 $\lambda_0$ 与模型可承载模式数 $M(N)$ 联系的结构性假设。本文给出一个最小且可证伪的 ''资源密度'' 假设 $\lambda_0(N)\propto N/M(N)$,从而推出
\[
\eta(N)\ \propto\ \frac{M(N)}{N}.
\]
若进一步假设容量映射 $M(N)\propto N^\gamma$,则得到幂律缩放 $\eta(N)\propto N^{\gamma-1}$;在常见情形 $\gamma=\tfrac12$ 时,得到 $\eta\propto N^{-1/2}$,等价于宽度倍率下的 $\eta\propto 1/\text{width}$ 级别缩放,与 $\mu$P 类方法在形式上相容。
\end{abstract}

\tableofcontents

\section{引言}
经验上,大模型训练往往需要随规模调整学习率等超参数;而 $\mu$P 一类工作强调:存在一套参数化与超参数缩放,使得学习率等可在不同宽度间近似稳定(超参数迁移)。本文的目标是:在你提出的 ''Modes--Zipf--隐式偏置'' 理论中,推导一个\textbf{学习率 $\eta$ 随模型规模 $N$ 的缩放律},并明确指出推导所需的最小额外假设与可检验预测。

\medskip
本文结构如下:\Cref{sec:modes} 给出 modes 与 Zipf 假设以及 excess loss 分解;\Cref{sec:dynamics} 给出 $q_k$ 的乘法收缩动力学并导出 ''内禀时间'';\Cref{sec:capacity} 给出容量映射 $M(N)$;\Cref{sec:lr-scaling} 在框架内推导学习率缩放律,并讨论与 $\mu$P 的形式对齐与可证伪点。

\section{Modes、Zipf 与 excess loss 分解}\label{sec:modes}

\begin{definition}[模式与频率分布]
设存在可数的模式集合 $\mathcal{K}=\{1,2,3,\dots\}$。在数据分布下,模式 $k$ 出现的概率为 $p_k$,满足 $\sum_{k\ge 1}p_k=1$。
\end{definition}

\begin{assumption}[Zipf 幂律分布]\label{ass:zipf}
存在指数 $\alpha>1$ 与缓变函数 $L(\cdot)$,使得当 $k$ 足够大时
\[
p_k \;=\; k^{-\alpha}L(k).
\]
为便于显式推导常数项,后文常采用纯幂律特例
\[
p_k \;=\; \frac{1}{Z}k^{-\alpha},\qquad Z=\sum_{k=1}^\infty k^{-\alpha}=\zeta(\alpha).
\]
\end{assumption}

\begin{definition}[excess loss 的模式分解]\label{def:excess}
设总体风险(或损失)为 $L$,不可约误差(Bayes 风险或熵项等)为 $E$,定义 excess loss
\[
\Delta L := L-E.
\]
假设 $\Delta L$ 可按模式加和分解为
\begin{equation}\label{eq:excess-sum}
\Delta L \;\approx\; \sum_{k=1}^\infty p_k\,q_k,
\end{equation}
其中 $q_k\in[0,1]$ 表示模型在模式 $k$ 上的 ''剩余误差''(未学习/未拟合程度)。
\end{definition}

\begin{remark}
\Cref{eq:excess-sum} 的含义是:模式越常见($p_k$ 越大),其剩余误差对总损失贡献越大;而隐式偏置将驱动训练优先压低高频模式的 $q_k$(见 \Cref{sec:dynamics})。
\end{remark}

\section{乘法收缩动力学与内禀时间}\label{sec:dynamics}

\subsection{有序学习与有效前沿}
\begin{assumption}[有序学习与有效前沿]\label{ass:ordered}
训练过程中存在一族随训练进度变化的剩余误差序列 $\{q_k(\tau)\}_{k\ge 1}$($\tau$ 为 token-step),满足:
\begin{enumerate}
\item (单调性)$0\le q_1(\tau)\le q_2(\tau)\le \cdots \le 1$;
\item (有效前沿)存在 $k_\star(\tau)$ 使得当 $k\ll k_\star(\tau)$ 时 $q_k(\tau)\to 0$,当 $k\gg k_\star(\tau)$ 时 $q_k(\tau)\to 1$。
\end{enumerate}
\end{assumption}

\begin{remark}
\Cref{ass:ordered} 是对 ''隐式偏置 + feature learning'' 的抽象刻画:训练更倾向先学习高频/低复杂度模式,形成由 $k_\star$ 划分的 ''学会'' 与 ''未学会'' 区域。
\end{remark}

\subsection{i.i.d.\ 采样与乘法更新}
\begin{assumption}[token-step 下的 i.i.d.\ 模式采样]\label{ass:iid}
每个 token-step $t=0,1,2,\dots$ 观测到的模式 $K_t\in\mathbb{N}$ 独立同分布,且
\[
\Pr[K_t=k]=p_k.
\]
记指示变量 $I_{t,k}:=\mathbf{1}\{K_t=k\}$,则 $\mathbb{E}[I_{t,k}]=p_k$。
\end{assumption}

\begin{assumption}[乘法收缩更新]\label{ass:mult}
对任意模式 $k$,其剩余误差 $q_k(t)\in[0,1]$ 在离散 token-step 下按如下方式更新:
\[
q_k(t+1)=
\begin{cases}
(1-\eta\lambda_k)\,q_k(t), & I_{t,k}=1,\\
q_k(t), & I_{t,k}=0,
\end{cases}
\]
其中 $\eta>0$ 为(全局)学习率,$\lambda_k>0$ 表示模式 $k$ 的有效学习强度。假设初始 $q_k(0)=1$。
\end{assumption}

\begin{assumption}[频率调制的有效学习强度]\label{ass:lambda}
存在常数 $\lambda_0>0$ 与指数 $\beta>0$,使得
\begin{equation}\label{eq:lambda-k}
\lambda_k \;=\; \lambda_0\,p_k^{\beta-1}.
\end{equation}
当 $\beta=1$ 时,各模式的 $\lambda_k$ 近似同阶;当 $\beta>1$ 时,高频模式具有更强的有效收缩强度;当 $0<\beta<1$ 时则相反。
\end{assumption}

\subsection{连续近似与显式解}
记截至 token-step $\tau$,模式 $k$ 被观测到的次数为
\[
n_k(\tau):=\sum_{t=0}^{\tau-1}I_{t,k}.
\]
由 \Cref{ass:mult} 立即得到
\begin{equation}\label{eq:q-discrete}
q_k(\tau) \;=\; (1-\eta\lambda_k)^{n_k(\tau)} q_k(0).
\end{equation}

\begin{proposition}[指数形式与内禀时间]\label{prop:exp}
在 ''小步长'' 条件 $\eta\lambda_k\ll 1$ 下,结合大数定律 $n_k(\tau)=\tau p_k+o(\tau)$,有近似
\begin{equation}\label{eq:q-exp}
q_k(\tau)\ \approx\ \exp\!\big(-\eta\lambda_k\,\tau p_k\big)
\ =\ \exp\!\big(-c\,\tau\,p_k^\beta\big),
\qquad c:=\eta\lambda_0.
\end{equation}
等价地,存在连续时间动力学
\begin{equation}\label{eq:ode}
\frac{d}{d\tau}q_k(\tau)=-c\,p_k^\beta q_k(\tau),\qquad q_k(0)=1.
\end{equation}
\end{proposition}

\begin{proof}
由 \Cref{eq:q-discrete} 与 $\log(1-x)=-x+o(x)$($x\to 0$)得
\[
(1-\eta\lambda_k)^{n_k(\tau)}
=\exp\!\big(n_k(\tau)\log(1-\eta\lambda_k)\big)
\approx \exp\!\big(-\eta\lambda_k\,n_k(\tau)\big).
\]
代入 $n_k(\tau)=\tau p_k+o(\tau)$ 并结合 \Cref{ass:lambda} 即得 \Cref{eq:q-exp}。对 \Cref{eq:q-exp} 求导即可得到 \Cref{eq:ode}。
\end{proof}

\begin{definition}[内禀时间尺度]\label{def:intrinsic-time}
定义内禀时间(或 ''有效训练时间'')
\[
s \;:=\; c\,\tau \;=\; \eta\lambda_0\,\tau.
\]
在 \Cref{prop:exp} 的近似下,$\{q_k\}$ 以及由其诱导的 $\Delta L$ 的训练动力学仅通过 $s$ 进入。
\end{definition}

\begin{remark}[学习率在该抽象层面的地位]
在 \Cref{def:intrinsic-time} 下,学习率 $\eta$ 的作用等价于对横轴(token-step)做线性重标尺:$\tau \mapsto s=\eta\lambda_0\,\tau$。因此,要讨论 ''$\eta$ 随模型规模 $N$ 如何缩放'',本质上是在讨论:当 $N$ 改变时,$\lambda_0$ 是否保持常数;若不保持,则应如何选择 $\eta(N)$ 使 $s$ 的标定保持可比。
\end{remark}

\subsection{训练前沿与时间缩放(复述框架中的结论)}
从 \Cref{eq:q-exp} 可见,模式 $k$ 何时 ''被学会'' 由量级条件 $c\tau p_k^\beta\gtrsim 1$ 控制。

\begin{definition}[训练前沿]\label{def:k-tau}
定义训练前沿 $k_\tau$(忽略常数)为满足
\begin{equation}\label{eq:frontier-condition}
c\tau\,p_{k_\tau}^\beta \asymp 1
\end{equation}
的模式指标。
\end{definition}

在纯幂律 $p_k=\frac{1}{Z}k^{-\alpha}$ 下,由 \Cref{eq:frontier-condition} 得到
\begin{equation}\label{eq:k-tau-scaling}
k_\tau \;\asymp\; Z^{-1/\alpha}\,(c\tau)^{1/(\alpha\beta)}.
\end{equation}

此外,将 \Cref{eq:q-exp} 代入 excess loss 分解 \Cref{eq:excess-sum},可得到时间/计算缩放的幂律(此处给出一个标准推导版本):

\begin{lemma}[时间缩放律与 $c$ 的幂次进入]\label{lem:time-scaling}
在纯幂律 $p_k=\frac{1}{Z}k^{-\alpha}$ 下,令
\[
\Delta L(\tau) \approx \sum_{k=1}^\infty p_k\exp(-c\tau p_k^\beta),
\]
则当 $\tau\to\infty$ 时有渐近
\begin{equation}\label{eq:time-scaling}
\Delta L(\tau)\ \sim\
\underbrace{\frac{1}{\alpha\beta}\,Z^{-1/\alpha}\,
\Gamma\!\Big(\frac{\alpha-1}{\alpha\beta}\Big)}_{=:C_{\alpha,\beta,Z}}
\cdot (c\tau)^{-(\alpha-1)/(\alpha\beta)}.
\end{equation}
\end{lemma}

\begin{proof}
用积分近似和变量替换。设 $p(x)=\frac{1}{Z}x^{-\alpha}$,则
\[
\Delta L(\tau)\approx \int_{1}^{\infty}\frac{1}{Z}x^{-\alpha}
\exp\!\Big(-c\tau\Big(\frac{1}{Z}x^{-\alpha}\Big)^\beta\Big)\,dx
=\int_1^\infty \frac{1}{Z}x^{-\alpha}\exp(-a\tau x^{-\alpha\beta})dx,
\]
其中 $a:=cZ^{-\beta}$。令 $u=a\tau x^{-\alpha\beta}$,则
$x=(a\tau/u)^{1/(\alpha\beta)}$,
$dx=-(1/(\alpha\beta))(a\tau)^{1/(\alpha\beta)}u^{-1/(\alpha\beta)-1}du$。
代入并整理幂次,可得
\[
\Delta L(\tau)\approx \frac{1}{Z}\cdot \frac{1}{\alpha\beta}
(a\tau)^{-(\alpha-1)/(\alpha\beta)}
\int_{0}^{a\tau} e^{-u}\,u^{(\alpha-1)/(\alpha\beta)-1}\,du.
\]
当 $\tau\to\infty$ 时,上限 $a\tau\to\infty$,积分趋于 Gamma 函数
$\Gamma\big(\frac{\alpha-1}{\alpha\beta}\big)$,从而
\[
\Delta L(\tau)\sim \frac{1}{Z}\cdot \frac{1}{\alpha\beta}
(cZ^{-\beta}\tau)^{-(\alpha-1)/(\alpha\beta)}
\Gamma\!\Big(\frac{\alpha-1}{\alpha\beta}\Big)
=\frac{1}{\alpha\beta}Z^{-1/\alpha}\Gamma\!\Big(\frac{\alpha-1}{\alpha\beta}\Big)
(c\tau)^{-(\alpha-1)/(\alpha\beta)}.
\]
\end{proof}

\begin{remark}
\Cref{lem:time-scaling} 直接显示:$c=\eta\lambda_0$ 以幂次形式进入时间缩放律。换言之,若我们希望不同设置下训练曲线的时间尺度可对齐,核心是控制 $c(N)$ 的变化。
\end{remark}

\section{容量映射与模型规模误差}\label{sec:capacity}

\begin{assumption}[容量映射(capacity map)]\label{ass:capacity}
存在函数 $M:\mathbb{N}\to\mathbb{N}$(随模型规模 $N$ 单调增长),使得当模型规模为 $N$ 时,有效前沿满足
\[
k_\star(N)\ \asymp\ M(N).
\]
并假设存在 $\gamma>0$ 使得幂律容量映射成立:
\[
M(N)\ \propto\ N^\gamma.
\]
\end{assumption}

\begin{proposition}[模型规模误差的幂律]\label{prop:model-scaling}
在 \Cref{ass:zipf,ass:capacity} 下,若训练足够充分使得 $k\lesssim M(N)$ 的模式被学习($q_k\approx 0$),则模型规模瓶颈下的剩余误差满足
\begin{equation}\label{eq:epsN}
\varepsilon_N(N)
\ :=\ \sum_{k>M(N)}p_k
\ \asymp\ M(N)^{-(\alpha-1)}
\ \asymp\ N^{-\gamma(\alpha-1)}.
\end{equation}
\end{proposition}

\begin{proof}
由 Zipf 幂律尾和(积分比较):
$\sum_{k>M}k^{-\alpha}\asymp \int_M^\infty x^{-\alpha}dx \asymp M^{-(\alpha-1)}$。
代入 $M=M(N)\propto N^\gamma$ 即得。
\end{proof}

\begin{remark}
\Cref{prop:model-scaling} 是 ''模型规模 scaling law'' 的一条直接来源:只要 tail(未覆盖模式)主导误差,就得到幂律指数 $\gamma(\alpha-1)$。
\end{remark}

\section{学习率随模型规模的缩放律}\label{sec:lr-scaling}

\subsection{为什么仅靠原框架无法推出 $\eta(N)$}
到目前为止,学习率 $\eta$ 只通过 $c=\eta\lambda_0$ 进入训练动力学;而模型规模 $N$ 仅通过 $M(N)$ 进入模型瓶颈误差 \Cref{eq:epsN}。若不说明 $\lambda_0$ 与 $N$ 的关系,则 \textbf{$N$ 与 $\eta$ 在理论上彼此独立}:任何 $\eta(N)$ 都与 \Cref{ass:capacity} 不发生代数耦合,因此无法推出唯一缩放律。

因此,要从该理论推出 ''$\eta$ 应如何随 $N$ 缩放'',必须补充一条把 $\lambda_0$ 连接到 $N$(或 $M(N)$)的结构性假设。下面给出一个最小且自然的选择。

\subsection{内禀时间不变性:缩放律的第一原则}

\begin{theorem}[尺度不变性原则:固定 $c(N)$]\label{thm:keep-c}
在 \Cref{ass:iid,ass:mult,ass:lambda} 下,对任意给定的模式频率 $\{p_k\}$,训练动力学由
\[
q_k(\tau)\approx \exp\big(-c(N)\tau p_k^\beta\big),\qquad c(N):=\eta(N)\lambda_0(N),
\]
决定。特别地,若希望不同模型规模 $N$ 下的训练曲线在 ''内禀时间'' 坐标 $s=c(N)\tau$ 上可直接对齐(即实现零样本超参数迁移意义下的动力学相似),充分条件是
\begin{equation}\label{eq:c-const}
c(N)\equiv c_\mathrm{ref}\quad\text{为常数}.
\end{equation}
\end{theorem}

\begin{proof}
由 \Cref{prop:exp,def:intrinsic-time},$q_k$ 仅依赖于 $s=c\tau$。若对所有 $N$ 取相同 $c_\mathrm{ref}$,则对任意 $\tau$,$q_k$ 与由其诱导的 $\Delta L$ 都是同一个函数在相同 $s$ 上的取值,从而实现动力学对齐。
\end{proof}

\begin{remark}[与稳定性条件的一致性]
小步长与稳定性通常要求 $\eta\lambda_k\ll 1$,而 $\lambda_k\le \lambda_0$(当 $\beta\ge 1$ 且 $p_k\le 1$)时,固定 $c=\eta\lambda_0$ 也等价于固定最坏情形的步长尺度,从而稳定性不随 $N$ 系统性恶化。
\end{remark}

\subsection{资源密度假设:把 $\lambda_0(N)$ 接到 $M(N)$}

\begin{assumption}[模式资源密度 $\Rightarrow$ 基准学习强度]\label{ass:resource}
将模型规模 $N$ 理解为可调自由度总量。容量映射 $M(N)$ 描述模型能稳定承载/表征的有效模式数量。假设在已覆盖模式集合上,\textbf{单个模式的平均资源密度}与 $N/M(N)$ 同阶,且基准有效学习强度 $\lambda_0(N)$ 与该资源密度成正比:
\begin{equation}\label{eq:lambda0-resource}
\lambda_0(N)\ \propto\ \frac{N}{M(N)}.
\end{equation}
\end{assumption}

\begin{remark}[可证伪性]
\Cref{ass:resource} 是本文唯一新增的 ''结构性'' 假设。它可通过实验估计 $\lambda_0(N)$ 的相对变化(例如从早期训练曲线拟合 $c(N)$,或从 $q_k$ 的衰减速度估计)来检验;若观测到 $\lambda_0(N)$ 近似常数,则学习率无需按本文方式缩放。
\end{remark}

\subsection{主结论:$\eta(N)\propto M(N)/N$}

\begin{theorem}[学习率--模型规模缩放律]\label{thm:lr-scaling}
在 \Cref{thm:keep-c,ass:resource} 下,若希望不同模型规模 $N$ 的训练动力学在内禀时间上保持相似(即 $c(N)$ 保持常数),则学习率应满足
\begin{equation}\label{eq:eta-scaling-main}
\eta(N)\ \propto\ \frac{1}{\lambda_0(N)}\ \propto\ \frac{M(N)}{N}.
\end{equation}
进一步地,若 $M(N)\propto N^\gamma$,则得到幂律缩放
\begin{equation}\label{eq:eta-power}
\eta(N)\ \propto\ N^{\gamma-1}.
\end{equation}
\end{theorem}

\begin{proof}
由 \Cref{thm:keep-c},要求 $c(N)=\eta(N)\lambda_0(N)\equiv c_\mathrm{ref}$。于是 $\eta(N)\propto 1/\lambda_0(N)$。再由 \Cref{ass:resource} 得到 \Cref{eq:eta-scaling-main};代入 $M(N)\propto N^\gamma$ 得 \Cref{eq:eta-power}。
\end{proof}

\begin{corollary}[宽度倍率形式与 $\mu$P 形式对齐]\label{cor:width}
设某类架构的参数量与宽度 $d$ 近似满足 $N\propto d^2$。若容量映射满足 $M(N)\propto \sqrt{N}\propto d$(即 $\gamma=\tfrac12$),则
\[
\eta(N)\propto N^{-1/2}\ \propto\ d^{-1}.
\]
即:宽度放大 $m$ 倍($d\mapsto md$)时,学习率按 $1/m$ 级别缩放。
\end{corollary}

\begin{remark}
\Cref{cor:width} 解释了为何在一些经验/工程规则中会出现 ''宽度放大 $m$ 倍,学习率缩小 $\sim 1/m$'' 的规律:在本文框架中,这对应于 ''可承载模式数与宽度线性增长'' 的容量映射,以及 ''基准学习强度与单模式资源密度正比'' 的结构假设。
\end{remark}

\subsection{进一步讨论:若 $\lambda_0(N)$ 的指数不同会怎样?}
为便于你在论文中扩展,给出一个一般化的注记。

\begin{remark}[一般化:$\lambda_0(N)\propto (N/M(N))^\delta$]
若把 \Cref{ass:resource} 推广为 $\lambda_0(N)\propto (N/M(N))^\delta$($\delta>0$),则在保持 $c(N)$ 常数的原则下有
\[
\eta(N)\propto \Big(\frac{M(N)}{N}\Big)^\delta.
\]
当 $M(N)\propto N^\gamma$ 时,得到 $\eta(N)\propto N^{\delta(\gamma-1)}$。
本文取 $\delta=1$ 是最小的 ''线性资源密度'' 情形。
\end{remark}

\section{结论}
本文在 Modes--Zipf--隐式偏置的抽象框架中,指出学习率只通过 $c=\eta\lambda_0$ 进入训练动力学,从而 ''学习率随模型规模缩放'' 的问题等价于:厘清 $\lambda_0$ 随 $N$ 的变化。通过补充一个最小的资源密度假设 $\lambda_0(N)\propto N/M(N)$,我们得到主结论
\[
\eta(N)\propto \frac{M(N)}{N}\propto N^{\gamma-1}.
\]
该缩放律的关键优势是可证伪:只要能从训练曲线或模式级残差衰减估计 $\lambda_0(N)$ 的相对变化,就可以直接验证/否定该结论,并据此修正 $\delta$ 或容量映射 $M(N)$ 的形式。

% \begin{thebibliography}{9}
% \bibitem{kaplan}
% Jared Kaplan et al. Scaling Laws for Neural Language Models. NeurIPS 2020.
% \bibitem{hoffmann}
% Jordan Hoffmann et al. Training Compute-Optimal Large Language Models. arXiv:2203.15556, 2022.
% \end{thebibliography}

\end{document}
